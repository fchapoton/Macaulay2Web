%
%  jsag
%
%  Created by franzi on 2013-03-24.
%  Copyright (c) 2013 __MyCompanyName__. All rights reserved.
%
\documentclass[twocolumn]{article}

% Use utf-8 encoding for foreign characters
\usepackage[utf8]{inputenc}

% Setup for fullpage use
\usepackage{fullpage, hyperref}

\usepackage{abstract}
% Uncomment some of the following if you use the features
%
% Running Headers and footers
%\usepackage{fancyhdr}

% Multipart figures
%\usepackage{subfigure}

% More symbols
%\usepackage{amsmath}
%\usepackage{amssymb}
%\usepackage{latexsym}

% Surround parts of graphics with box
\usepackage{boxedminipage}

% Package for including code in the document
\usepackage{listings}

% If you want to generate a toc for each chapter (use with book)
\usepackage{minitoc}

% This is now the recommended way for checking for PDFLaTeX:
\usepackage{ifpdf}

%\newif\ifpdf
%\ifx\pdfoutput\undefined
%\pdffalse % we are not running PDFLaTeX
%\else
%\pdfoutput=1 % we are running PDFLaTeX
%\pdftrue
%\fi

\ifpdf
\usepackage[pdftex]{graphicx}
\else
\usepackage{graphicx}
\fi

\def\trym2{{\it TryM2}}
\def\M2{{\it Macaulay2}}

\title{\trym2, a Web Application for \M2}
\author{Lars Kastner\\ Technische Universit\"at Berlin \\{\small kastner\char`\@math.tu-berlin.de} \and
Franziska Hinkelmann\\Google Germany \\{\small franziska.hinkelmann\char`\@gmail.com} \and 
Michael Stillman\\Cornell University \\{\small mike\char`\@math.cornell.edu} \thanks{Stillman has been supported by NSF grants DMS 08-10909 and 10-02210. Hinkelmann has been supported by NSF award 0635561. Kastner by DFG SPP 1489. 
} }


\date{}

\begin{document}



\ifpdf
\DeclareGraphicsExtensions{.pdf, .jpg, .tif}
\else
\DeclareGraphicsExtensions{.eps, .jpg}
\fi


\twocolumn[
    \maketitle
        \begin{onecolabstract}
            \M2 is a software system devoted to supporting research in
            algebraic geometry and commutative algebra, whose creation
            has been funded by the National Science Foundation since
            1992. It is a command-line tool that has no graphical user
            interface.  This manuscript describes a new web
            application, named \trym2, interfacing to \M2. It requires no
            installation and provides a graphical user
            interface, thereby making \M2 more accessible to
            first time users and students.  \trym2
            has all features that the desktop version
            offers.  In addition, it contains tutorials that explain
            different concepts of algebraic geometry such as Gr\"obner
            bases. Users can also create their own tutorials.  \trym2
            has been used in courses at Cornell University,
            Harvard University, Georgia Tech, and Free University of
            Berlin.
        \end{onecolabstract}
]
\saythanks
\section{Introduction}

\M2 is a software system devoted to supporting research in algebraic
geometry and commutative algebra, whose creation and development have
been funded by the National Science Foundation since 1992~\cite{M2}.
This article describes a new web application, named \trym2,
interfacing to \M2.  It is available at~\cite{webM2}.  This web application
has the advantage that users do not need to download or install any
software. A browser and internet connection are enough to run
calculations in \M2. It has the same functionality as the desktop
version, albeit users might have access to less resources than on
their own machine. Users can upload files, load packages, and generate
files such as images.

Our primary motivation for creating this web application was to
provide an easy to use experience for classroom and student use.
Having students download and install software such as \M2 is time
consuming and inevitably there are some situations for which this
process fails or takes significant effort to get running.
\trym2 is designed to
be able to handle many students at once. To keep it as simple as possible,
there is no sign-up or login needed.

\trym2 was used at the Syzygies
meeting in Berlin in May 2013, with about 70 users. It has been used in many courses, including at Cornell
University, Harvard University, Georgia Tech, and Free University of
Berlin. \trym2 is suitable for both beginners and seasoned experts.

We considered off the shelf solutions, such as the Sage
notebook~\cite{sagenotebook}, but we wanted a more
lightweight solution, in particular one that does not require users to create
new accounts at a web site.

\trym2 runs in modern browsers, including Chrome, Firefox, Safari, and Edge, both on mobile devices and desktop
computers.

\trym2 provides several tutorials. Users can create their own
tutorials and share them with other \M2 users.

\begin{figure*}[htb]
    \includegraphics[width=.99\textwidth]{homeWebsite.jpg}
    \caption{A typical view of \trym2. The left hand
        side shows a tutorial giving an introduction to Gr\"obner
        bases. Text in yellow boxes can be clicked and is then executed by
        \M2 on the server. The complete output of the calculations
        is show on the right hand side. Clicking {\it Home, Tutorial, Editor, About}
        changes the left hand view, {\it Reset, Interrupt}
        reset and interrupt the \M2 session on the server, {\it
        Save} provides both the input and the output of the current
        session to the user as a text file, and {\it Upload File} uploads files
        that can then be accessed by \M2.}
\label{fig:home}
\end{figure*}

In Section 2, we describe the basics of using \trym2.
Section 3 details how to create tutorials.
Providing a program such as \M2 which allows users to
access system resources presents a number of challenges to keep users
from naively or mischievously misusing the system.  In Section 4, we
outline technical details of the implementation of \trym2.

\trym2 is open source, and available on GitHub~\cite{github}.  The
\trym2 server can be run on personal laptops and computers,
institutional servers, or in the cloud, e.g. Amazon Web Services.
Details are provided at \cite{github}.  If you would like more
information on running your own \trym2 instance or using \trym2 in a
course, please contact one of us.  The open source backend
architecture is not specific to \M2.  It can be adapted to any
software with a command line interface.  A version for Singular
is already available online (\cite{trySingular}).

\section{Using \trym2}

An important design decision is to keep the interface as simple as
possible.  Therefore there is no login or registration.

After first opening \cite{webM2}, the user is presented with a split
window~\ref{fig:home}.  The right hand side of this window provides a
shell like environment running \M2. The user can type into it and use
the arrow keys for navigating through the command history.
The tab key works for command completion.

On the left hand side of the window, the user can switch between {\it Home}, {\it
  Tutorial}, {\it Editor}, and {\it About}. {\it Home} shows the available
 tutorials. {\it Tutorial} shows the currently selected
tutorial. Tutorials are interactive and contain executable pieces of
\M2 code that are run by clicking on them. {\it Editor} is a
text area in which one can type \M2 commands and evaluate them.

All code executed, either by clicking on interactive parts in a
tutorial or by entering code in the {\it Editor} window, will appear in 
the shell emulator on the right hand side together with the \M2 output.

The {\it Reset} button resets a \M2 session, i.e., restarts 
the \M2 process: it stops any running calculation, deletes all variables, 
unloads all packages, and then reloads the standard packages.

The {\it Interrupt} button stops a running calculation, without
resetting the \M2 session. This can be used to stop a long running
calculation.

\subsection{Features}

The user can run any command in \trym2 that can be run in the desktop
computer version, including commands that refer to external programs
distributed with \M2, such as gfan, bertini, and phcpack.  \M2 comes
with a large number of contributed \M2 packages.  \trym2 can access
all of these packages.  As in the desktop computer version, one loads
such a \M2 package using {\tt needsPackage}, e.g. {\tt needsPackage
  "BoijSoederberg"}.  The user can upload files. For instance, \M2
packages not included in the \M2 distribution can be uploaded and then
loaded into the user's \M2 session using {\tt needsPackage}.  The {\it
  Upload File} button uploads files to the server where they are
available to the user's \M2 session.  This works for all file types,
including \M2 package files.

\begin{figure*}[htb]
    \includegraphics[width=.95\textwidth]{withGraph.jpg}
    \caption{Screenshot after the user
      has generated an image. Some \M2 packages can generate
      graphs. By invoking the command {\tt displayGraph}, the image is
      generated on the server and displayed to the user.}
    \label{fig:graph}
\end{figure*}

Results from a session can be retrieved by using the {\it Save}
button, or by using copy and paste. If \M2 generates
images such as graphs, those are presented to the user, see Figure \ref{fig:graph}.

There is a maximum number of concurrent users.  If there are not this
many users, the \M2 sessions are usually kept alive for several
days. When a user starts a \trym2 session and visits the web app at a
later time, they will continue the previous session. During times of
high demand, inactive sessions may be kept for a shorter period of
time, to make room for new users.  Occasionally the server must be
rebooted, this will unfortunately remove all active sessions.
Resources are generally more restricted than on a personal computer,
since multiple users are sharing these resources.

\section{Tutorials}

The purpose of the mathematical tutorials in \trym2 is to make abstract
mathematical concepts more concrete as well as to provide a playground for users
and students to experiment and explore the mathematics.

The tutorials are interactive, \M2 examples in a tutorial can be run
by clicking or pressing on the highlighted code.  The text of the
tutorial includes mathematics, using the mathjax tex library for
browsers.  Users can easily modify this \M2 code in the Editor, in
order to work examples and exercises, or to investigate further the
mathematical concepts taught in the tutorial.

\trym2 includes several tutorials describing basic \M2 usage and
mathematics around Gr\"obner bases and polynomial algebra.  These
tutorials include content written by the authors and by David
Eisenbud.

Additional tutorials can be used in \trym2 by simply uploading them in the
{\it Load your own tutorial} section.

\subsection{Authoring a tutorial}

The format of tutorials for \trym2 is the simple markdown format, used
on many websites, such as github. The title line begins with a ``\#''
and contains the title (and possibly author) of the tutorial. Each lesson
in the tutorial is marked by two ``\#''s, and each piece of \M2 code is enclosed in triple
backticks ``` (these should be on their own line).  The text itself
can contain latex math, as well as html markup.  Blank lines separate
paragraphs. A tutorial template file is available for download in the
{\it Load your own tutorial} section of \trym2, see
Figure~\ref{fig:markdown}.

Instructors or users who wish to share their tutorials with their
students or others should distribute their tutorials to them, e.g. via
email, a course website, dropbox, google drive, etc.  Each
user then uploads the tutorial in their own \trym2.  Note that an
uploaded tuorial is only available to that user, it is not made
publically available.  We encourage interested authors to share their
tutorials at the \trym2 google group (\cite{trym2:googlegroup}).  This
is also an excellent place to ask questions and get help about writing
tutorials.

Existing \M2 documentation in simpledoc format can be converted into
tutorials via the \M2 package {\tt TryM2Tutorials}.  This is how several of
the distributed tutorials were generated.

We are collaborating with several mathematicians to develop more
tutorials.  We will keep a curated collection of tutorials on various
mathematical topics at the \trym2 google group (\cite{trym2:googlegroup}).
 
\section{Internal structure}

The front end of \trym2 is written in TypeScript, and uses Material Design Lite
\cite{MDL} for user interface elements.  The server back end is
implemented in Node.js. The Node.js server functions as a bridge
between \M2 instances and users.  For every user, the server starts
a new \M2 process in a separate and sandboxed linux virtual environment.

Technically, this is achieved by using Docker
containers~\cite{docker}. Docker implements a high-level API to
provide lightweight containers that run processes in isolation.  The
server starts a new Docker container running \M2 for every user. The
Node.js server communicates with Docker containers via {\it ssh}. This
allows containers to run on independent machines, so in the future
this will allow the configuration of systems which can scale with
demand.

This architecture provides a sandboxed and secure linux environment
for every \trym2 user.  For instance, users can interact with the
sandboxed file system and run shell commands via \M2's {\tt get}
command:

{\tiny\begin{verbatim}
i1 : -- get information on the underlying
     -- operating system
     get "!uname -mrs" 
o1 = Linux 4.4.0-81-generic x86_64
\end{verbatim}
}

{\tiny\begin{verbatim}
i2 : -- write to a file     
     "results.txt" << "Hello!" << close
o2 = results.txt
o2 : File

i3 : -- list all files in the current
     -- directory
     get "!ls"
o3 = results.txt

i4 : -- obtain the file
     get "!open results.txt"
\end{verbatim}
}


To ease the setup process we provide a virtual machine that contains both the Node.js server
and Docker. This virtual machine is configured using {\tt vagrant}, a tool to create and
configure reproducible and portable development environments~\cite{vagrant}.

\section{Conclusion}

\trym2 lowers the entry barrier for new users of \M2, therefore increasing the number 
 of users and fostering research in computational algebraic geometry and commutative algebra. 

For all users, \trym2 is a useful way to use \M2 on shared or mobile
devices where it is difficult or impossible to install software. For
some, the user interface of \trym2 is a more natural way to interact with \M2.

\section{Acknowledgments}

We would like to thank Dan Grayson, Greg Smith, and Benjamin Lorenz for
fruitful discussions on system security and on user interface design.
We would like to thank David Eisenbud for writing his \M2 tutorials.


\bibliographystyle{plain}
\bibliography{references}
\end{document}
